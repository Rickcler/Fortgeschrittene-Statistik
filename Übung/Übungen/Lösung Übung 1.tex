\documentclass{article}

% Math and symbols packages
\usepackage{amsmath}
\usepackage{amssymb}
\usepackage{mathtools}
\usepackage{cancel}
% Formatting and layout
\usepackage[margin=1in]{geometry}
\usepackage{enumitem}
\usepackage{xcolor}
\usepackage{enumitem}

% Begin document
\begin{document}
\title{Übung 1 Lösungen}
\author{Victor Minig}
\date{\today}
\maketitle

\section*{1.}
\textbf{Gegeben:} \\

$M = 3$ verschiedene Kugeln die mit den Zahlen $1$ bis $3$ versehen sind. Es werden $n=2$ Kugeln gezogen. 
\subsection*{   a)} 

    Anzahl der \textit{Elementareignisse L} des \textit{Stichprobenraums S} beim Ziehen "mit Zurücklegen":

\begin{align*}
    &\{(1,1), (1,2),(1,3), (2,1), (2,2), (2,3), (3,1), (3,2), (3,3)\} \\
    & \Longrightarrow 9 ~ \text{Ergebnisse}
\end{align*}

und "ohne Zurücklegen":
\begin{align*}
    &\{(1,2),(1,3), (2,1), (2,3), (3,1), (3,2)\} \\
    & \Longrightarrow 6 ~ \text{Ergebnisse}
\end{align*}
\subsection*{   b)}
    Anzahl der Teilmengen des Stichprobenraums beim "Ziehen mit Zurücklegen"
\[
    2^{9} =  512
\]

\section*{2.}
\textbf{Gegeben:} \\

Ein Mengenkörper $\mathcal{K}$ ist durch die Eigenschaften definiert: 
\begin{enumerate}
    \item $A \in \mathcal{K} \Rightarrow \bar{A} \in \mathcal{K}$
    \item  $A \in \mathcal{K}$ und $B \in \mathcal{K} \Rightarrow A \cup B \in \mathcal{K}$
\end{enumerate}

Die Ergebnismenge $S = \{(i,j) : i, j = 1,2, \ldots, 6\}$ ist der Stichprobenraum des Experiments: Zweimaliges Ausspielen eines Würfels. $A_{k} ~ (k= 2, \ldots, 12)$ sei das Ereignnis: Die Augensume beider Ausspielungen ist kleiner oder gleich k und $B_{k}(k = 2, \ldots, 12)$ das Ereignis: Die Augensumme beider Ausspielungen ist größer als $k$ 
\\
Welche Teilmengen der Potenzmenge von $S$ bilden einen Mengenkörper?

\subsection*{a)}
\begin{align*}
    &\mathcal{K}_{1} = \{\emptyset, A_{2}, B_{2}, S\} \\
    &\Rightarrow \text{Ist Mengenkörper, da } \\
    & \qquad A_{2} = \bar{B_{2}},~  B = \bar{A_{2}},~ \emptyset = \bar{S},~ S = \bar{\emptyset}\\
    &\text{und} \\
    & \qquad A_{2} \cup B_{2} = S,~ A_{2} \cup S = S,~ B_{2} \cup S = S,~ A_{2} \cup \emptyset = A_{2},~ B_{2} \cup \emptyset = B_{2},~ S \cup \emptyset = S
\end{align*}
\subsection*{b)}
\begin{align*}
    &\mathcal{K}_{2} = \{A_{12}, B_{12}\} \\
    &\Rightarrow \text{Ist Mengenkörper, da } \\
    &\text{und} \\
    & \qquad \bar{A}_{12} = B_{12}, \bar{B}_{12} = A_{12}\\
    &\qquad A_{12} \cup B_{12} = A_{12} = S
\end{align*}
\subsection*{c)}
\begin{align*}
    &\mathcal{K}_{3} = \{A_{11}, B_{11}\} \\
    &\Rightarrow \text{Ist kein Mengenkörper, da  zwar} \\
    & \qquad \bar{A}_{11} = B_{11}, \bar{B}_{11} = A_{11}\\
    &\text{aber} \\
    &\qquad A_{11} \cup B_{11} = S \notin \mathcal{K}_{3}
\end{align*}
\subsection*{d)}
\begin{align*}
    &\mathcal{K}_{4} = \{A_{k}, B_{k}: k = 2, \ldots, 12 \} \\
    &\Rightarrow \text{Ist kein Mengenkörper, da  zwar} \\
    & \qquad \bar{A}_{k} = B_{k} \in \mathcal{K}_{4} \text{ und } \bar{B}_{k} = A_{k} \in \mathcal{K}_{4} \qquad \forall k \in \{ 1,\ldots,12\}\\
    &\text{aber} \\
    &\qquad \text{verschiedene Teilmengen nicht enthalten sind, wie z.B. } A_{2} \cup B_{11} = \{2, 12\} \notin \mathcal{K}_{4} 
\end{align*}
\section*{3. }
\textbf{Gegeben:}\\ 

$S = \{1,2,3\}$ Gesucht sind sämtliche nicht leere Teilmengen der Potenzmenge von $S$ die einen Mengenkörper bilden. 

\begin{align*}
    \{(1,2,3), \emptyset, (1), (2,3), (2), (1,3), (3), (1,2)\}\\
    \{(1,2,3), \emptyset, (1), (2,3)\} \\
    \{(1,2,3), \emptyset, (2), (1,3)\} \\
    \{(1,2,3), \emptyset, (3), (1,2)\} \\
    \{(1,2,3), \emptyset\}
\end{align*}

\section*{4. } 


Für jede der Folgenden Kombinationen von Stichprobenraum S, Ereignisraum $\mathcal{Y}$ und Mengenfunktion ist zu überprüfen,
ob die Funktion auch eine Wahrscheinlichkeitsmengenfunktion ist. \\

Definition einer Wahrscheinlichkeitsmengenfunktion P(A):
\begin{enumerate}[label=\roman*.]
    \item $\text{For } A \subset S, P(A) \geq 0 $
    \item $P(S) = 1$
    \item Let $I$ be a finite or countably infite index set of positive integers, and let $\{A_{i}: i \in I\}$ be a collection \\
     of disjoint events contained in $S$. Then, $P(\cup_{i\in I}A_{i}) = \sum_{i \in I} P(A_{i})$
\end{enumerate}

\subsection*{a)}
\textbf{Gegeben:}


\begin{enumerate}
    \item $S = \{1,2,3,4,5,6,7,8\}$ 
    \item $\mathcal{Y} = \{A: A \subset S\}$ 
    \item $P(A) = \sum_{x\in A} x / 36$ für $A \in \mathcal{Y}$
\end{enumerate}
\textbf{Lösung:}


\begin{enumerate}[label=zu \roman*.]
    \item Da $S$ definiert ist als eine Menge mit ausschließlich positiven Zahlen und und für jedes Event $A$ gilt, dass $A \subset S$, folgt dass \[P(A) = \sum_{x\in A} x / 36 \geq 0 \text{ für }A \in \mathcal{Y}\]
    \item $P(S) = \sum_{x \in S} x/36 = \sum_{1}^{8}x/36 = 1$
    \item If $I$ is a finite or countably infite index set of positive integers and $\{A_{i}: i \in I\}$ is a collection of disjoint events contained in S then it holds that \[P(\cup_{i\in I}A_{i}) = \sum_{x_{A_{i}} \in \cup_{i\in I}A_{i}}x_{A_{i}}/36 \overset{\text{Kommutativität}}{=} \sum_{x \in A_{1}} x/36 + \sum_{x \in A_{2}} x/36 + \ldots = \sum_{i\in I}P(A_{i})\]  
\end{enumerate}

$\Longrightarrow$ Dementsprechend ist $P$ eine Wahrscheinlichkeitsmengenfunktion

\subsection*{b)}
\textbf{Gegeben:}


\begin{enumerate}
    \item $S = [0,\infty)$ 
    \item $\mathcal{Y} = \{A: A \text{ ist ein Telinterval von  } S $ oder eine beliebige Menge gebildet aus Vereinigungen, Schnittmengen, oder Komplementmengen dieser Teilintervalle\}
    \item $P(A) = \int_{x\in A} e^{-x} dx$ für $A \in \mathcal{Y}$
\end{enumerate}
\textbf{Lösung:}


\begin{enumerate}[label=zu \roman*.]
    \item Für beliebiges $A \in \mathcal{Y}$ gilt, dass es direkt ein Teilintervall von $S$ ist, oder eine Menge, die sich aus Schnittmengen, Vereinigungen und Komplemntmengen eben dieser Teilintervalle zusammensetzt. Für jedes einzelne Teilintervall $(a, b) \text{ mit } 0 \leq a \leq b < \infty$ gilt: \[P((a,b))= \int_{a}^{b}e^{-x}dx = [-e^{-x}]^{b}_{a} = -e^{-b} -(-e^{-a})=-e^{-b} + e^{-a} \geq 0\] Da für jedes einzelne Interval $a,b$ gilt, dass $P((a,b))\geq 0 $ gilt dies automatisch auch für alle möglichen Vereinigungen, Schnittmengen und Komplementmengen.
    \item $P(S) = P([0,\infty)) = \int_{0}^{\infty} e^{-x} dx  = [-e^{-x}]_{0}^{\infty} = -e^{-\infty} -(-e^{0}) = 0 + 1 = 1$
    \item If $I$ is a finite or countably infite index set of positive integers and $\{A_{i}: i \in I\}$ is a collection of disjoint events contained in S then it holds that \[P(\cup_{i \in I}A_{i}) = \int_{x\cup_{i \in I}A_{i}} e^{-x} dx \overset{Lebesgue Integral}{=} \int_{x \in A_{1}} e^{-x} dx + \int_{x \in A_{2}} e^{-x} dx + \ldots = \sum_{i \in I} \int_{x \in A_{i}} e^{-x} dx = \sum_{i \in I} P(A_{i})\]   
\end{enumerate}

$\Longrightarrow$ Dementsprechend ist $P$ eine Wahrscheinlichkeitsmengenfunktion

\subsection*{c)}
\textbf{Gegeben:}


\begin{enumerate}
    \item $S = \{x: x \text{ ist eine positive ganze Zahl}\}$ 
    \item $\mathcal{Y} = \{A: A \subset S\}$
    \item $P(A) = \sum_{x \in A} x^{2}/10^{5}$
\end{enumerate}
\textbf{Lösung:}


\begin{enumerate}[label=zu \roman*.]
    \item Da jede Ereignis $A$ nur aus positiven ganzen Zahlen bestehen kann, und $x^{2} / 10^{5} > 0 ~ \forall x \in \mathbb{N}$, gilt \[ P(A) \geq 0 \qquad \forall A\in \mathcal{Y}\]
    \item Da $10^{5} \in S$, aber für $A = \{10^{5}\} \subset S$ gilt, dass $P(A) = (10^{5})^{2}/ 10^{5} = 10^{5} > 1$ muss gelten \[P(S) \neq 1\] 
\end{enumerate}

$\Longrightarrow$ Dementsprechend ist $P$ keine Wahrscheinlichkeitsmengenfunktion

\section*{5.}

\textbf{Gegeben:} \\

$P(\cdot)$ ist eine Wahrscheinlichkeitsmengenfunktion für den Ereignisraum $\mathcal{Y}$ mit $A_{i} \in \mathcal{Y} (i = 1, \ldots , r)$. 
\subsection*{a)}
\textbf{Zu zeigen:}\\ 

$P(A_{1} - A_{2}) = P(A_{1}) - P(A_{1} \cap A_{2})$\\ \\
\textbf{Lösung:}\\

Eine andere Schreibweise für $A_{1} - A_{2} \text{ ist } A_{1} \cap \bar{A}_{2}$. Es folgt:  \[P(A_{1} - A_{2}) = \overbrace{P(A_{1} \cap \bar{A}_{2}) + P(A_{1}\cap A_{2})}^{\qquad ~ ~P(A)\quad \text{Theorem 4}} - P(A_{1}\cap A_{2}) = P(A)- P(A_{1}\cap A_{2})\]
\subsection*{b)}
\textbf{Zu zeigen: }\\

$A_{i} \cap A_{j} = \emptyset \Rightarrow P(\cup_{i=1}^{r} A_{i}) = \sum_{i = 1}^{r} P(A_{i}) \qquad \forall i\neq j$ \\ \\
\textbf{Lösung:} \\

Komische Induktionsaufgabe. Wann anders anschauen
\subsection*{c)}
\textbf{Zu zeigen: }\\

$P([A_{1} \cup A_{2}] - [A_{1}\cap A_{2}]) = P(A_{1}) + P(A_{2}) - 2P(A_{1}\cap A_{2})$ \\ \\
\textbf{Lösung:} 
\begin{align*}
    P([A_{1} \cup A_{2}] - [A_{1}\cap A_{2}]) &= P([A_{1} \cup A_{2}] \cap \overline{[A_{1} \cap A_{2}]}) \tag*{$(A\cup B ) \cap C = (A\cap C) \cup (B \cap C)$}\\
    &=P([A_{1}\cap \overline{[A_{1} \cap A_{2}]}] \cup [A_{2}\cap \overline{[A_{1} \cap A_{2}]}] ) \tag*{$A \cap \overline{A\cap B}= A \cap \overline{B}$}\\
    &= P([A_{1}\cap \overline{A}_{2}] \cup [A_{2}\cap \overline{A}_{1}]) \tag*{$[A_{1}\cap \overline{A}_{2}] \cap [A_{2}\cap \overline{A}_{1}] = \emptyset$} \\
    &= P([A_{1}\cap \overline{A}_{2}]) + P([A_{2}\cap \overline{A}_{1}]) \tag*{A = A + B - B}\\
    &= \overbrace{P([A_{1}\cap \overline{A}_{2}]) + P([A_{1}\cap A_{2}])}^{P(A_{1})} - P([A_{1}\cap A_{2}]) + \overbrace{P([A_{2}\cap \overline{A}_{1}]) + P([A_{1}\cap A_{2}])}^{P(A_{2})} - P([A_{1}\cap A_{2}]) \\
    &= P(A_{1}) + P(A_{2}) - 2P(A_{1}\cap A_{2})
\end{align*}

\section*{6.}

\textbf{Gegeben:} \\

$A, B, C$ sind Elemente des Ereignisraums $\mathcal{Y}$ und $P(\cdot)$ eine entsprechende Wahrscheinlichkeitsmengenfunktion.

\subsection*{a)}

\textbf{Zu zeigen: } \\

$P(A) \leq P(B) \Rightarrow P(\overline{B}) \leq P(\overline{A})$   \\ \\
\textbf{Lösung:} \\

Da für jedes Element $A \in \mathcal{Y}$ gilt, dass $A \cup \overline{A} = S$, und da desweiteren gilt, dass $A \cap \overline{A} = \emptyset$, folgt, dass $P(S) = 1 = P(A) + P(\overline{A})$ und damit $P(\overline{A}) = 1 - P(A)$. Daraus folgt:   
\begin{align*}
    P(A) &\leq P(B) \\
    \Rightarrow  1-P(\overline{A}) &\leq 1 - P(\overline{B}) \tag*{$-1$}\\
    \Rightarrow -P(\overline{A}) &\leq - P(\overline{B}) \tag*{$\cdot (-1)$; Dreht das Ungleichunssymbol um} \\
    \Rightarrow P(\overline{A}) &\geq P(\overline{B})  
\end{align*}

\subsection*{a)}

\textbf{Zu zeigen: } \\

$A \subset B \Rightarrow \overline{B} \subset \overline{A}$ \\ \\
\textbf{Lösung:} \\

Da $A \subset B$ und $A \cup \overline{A} = \mathcal{Y}$  gilt, dass $\mathcal{Y} - B = \overline{B} $


\section*{7.}
\textbf{Gegeben:}

$P(A) = 3/4$ und $P(B) = 3/8$

\subsection*{a)}


\textbf{Zu zeigen:} \\

$P(A \cup B) \geq 3/4$ \\ \\
\textbf{Lösung:}  

\begin{align*}
    P(A \cup B) &= P(A \cup (\overline{A} \cap B)) \tag*{Mengen haben keine Schnittmenge}\\
    &=b P(A) + P(\overline{A} \cap B) \\
    &= 3/4 + P(\overline{A} \cap B) \tag*{$ \overline{A}\cap B  \subset S \Rightarrow P(\overline{A}\cap B) \geq 0$} \\
    &\geq 3/4
\end{align*}


\subsection*{b)}


\textbf{Zu zeigen:} \\

$1/8 \leq P(A\cap B) \leq 3/8$ \\ \\
\textbf{Lösung:}  \\

Da $P(S) =1$ und $P(A) + P(B) = 3/4 + 3/8 = 9/8 > 1$ muss $P(A\cap B) \geq 9/8 - 1 = 1/8$ sein \\

Außerdem, da $P(B) \geq P(A \cap B )$ sein muss (da $A \cap B \subset ((A \cap B) \cup (\overline{A} \cap B)) = B $), gilt $P(A\cap B) \leq 3/8$

\section*{8}

\textbf{Gegeben:}\\

Urne mit $M$ Kugeln, $n$ Ziehungen mit Zurücklegen. \\ \\
\textbf{Gesucht:} \\

$P(A)$ wobei $A$: Mindestens eine Kugel wird mehr als einmal gezogen. \\ \\
\textbf{Lösung:} \\ 

$P(A) = 1 - P(\overline{A})$ wobei $\overline{A}$: alle Kugeln sind unterschiedlich. Bei $M$ Kugeln 

\section*{9}


\textbf{Gesucht:} \\


$P(A)$ wobei $A$:  2 Asse us einem Kartenspeil mit 52 Karten ziehen (ohne Zurücklegen)
\textbf{Lösung:} \\

Die Wahrscheinlichkeit ein Ass aus einem Kartenspiel mit 52 Karten zu ziehen beträgt $4/52$ und die Wahrscheinlichkeit aus dem daraus resultierenden, nur noch 51 Karten umfassenden Kartendeck ein Ass zu ziehen beträgt $3/51$. Insofern gilt: \[ P(A) = 4/52 \cdot 3/51 = 12/2652= 3/663\]
 

\section*{10}


\textbf{Gegeben:} \\

$5$ Urnen, nummeriert von $1$ bis $5$. Jede Urne enthält $10$ Kugeln. Die Urne $i ~ (i = 1,...,5)$ enthält $i$ schwarze 10 - $i$ rote Kugeln. Das Zufallsexperiment ist, das zuerst zufällig eine Urne ausgewählt wird und dann eine Kugel
\subsection{a)}

\textbf{Gesucht:} \\

$P(S)$ wobei $S$: Eine schwarze Kugel wird gezogen:\\ \\
\textbf{Lösung:} \\

Jede Urne $U_{i}$ hat eine Chance von $P(U_{i})=1/5$ genommen zu werden. Wenn $S$ das Ereignis ist ist eine schwarze Kugel zu ziehen lässt es sich berechnen durch 
\begin{align*}
    P(S) &= \sum_{i = 1}^{5} P(S \cap U_{i}) \\
    &= \sum_{i =1}^{5} P(U_{i}) \cdot P(S|U_{i}) \\
    &= \sum_{i = 1}^{5} 1/5 \cdot i/10 \\
    &= 1/5 \cdot \sum_{i = 1}^{5} i/10 \\ 
    &= 1/5 \cdot 15/10 \\
    &= 3/10
\end{align*}  

\subsection*{b)}

\textbf{Gesucht:}\\

Unter der Gegebenheit, dass eine schwarze Kugel gezogen wurde, wie groß ist die Wahrscheinlichkeit das jene aus Urne 5 stammt.\\ \\

\textbf{Lösung:}\\ 

Gesucht ist also $P(U_{5}|S)$. Laut Bayes lässt sich für zwei Ereignisse $A,~B$ die bedingte Wahrscheinlichkeit $P(A|B)$ berechnen durch $P(A|B) = \frac{P(B|A)\cdot P(A)}{P(B)}$. Es folgt:
\begin{align*}
    P(U_{5}|S) &= \frac{P(S|U_{5})\cdot P(U_{5})}{P(S)}& \\
    &= \frac{(1/2) \cdot (1/5)}{3/10}&= 1/3
\end{align*}


\section*{16}

\textbf{Gegeben:} \\

$P(A_{i} = 0.1)$ für $i = 1, \ldots, 10$ \\ \\

\textbf{Gesucht:} \\

$P(\cap_{i=1}^{n})$ wenn ...\\

\subsection*{a)}

\textbf{Gegeben:} \\

Die $A_{i}$'s unabhängig sind \\ \\
\textbf{Lösung:} \\

Wenn die $A_{i}$'s unabhängig sind gilt: 

\begin{align*}
    P(\cap_{i=1}^{1}A_{i}) &= \prod_{i =1}^{10} P(A_{i}) \\
    &= \prod_{i =1}^{10} 0.1 \\
    &= 0.1^{10}
\end{align*}

\subsection*{b)}

\textbf{Gegeben:} \\

Die $A_{i}$'s disjunkt sind \\ \\
\textbf{Lösung:} \\

Wenn die $A_{i}$'s unabhängig sind gilt  $\cap_{i=1}^{1}A_{i} = \emptyset$. Dementsprechend: \[P(\cap_{i=1}^{1}A_{i}) = 0\]
\\ \\ \\
\subsection*{c)}

\textbf{Gegeben:} \\

$P(A_{i}|\cap_{j=1}^{i-1}) = 0.2 $ für $i = 2, \ldots, 10$ \\ \\
\textbf{Lösung:} \\

Für $i = 10$ gilt $P(A_{10}| \cap_{j=1}^{9}) = 0.2$


\end{document}