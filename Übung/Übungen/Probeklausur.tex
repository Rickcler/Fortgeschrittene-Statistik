\documentclass{article}

% Math and symbols packages
\usepackage{amsmath}
\usepackage{amssymb}
\usepackage{mathtools}
\usepackage{cancel}
% Formatting and layout
\usepackage[margin=1in]{geometry}
\usepackage{enumitem}
\usepackage{xcolor}
\usepackage{enumitem}


% Math and symbols packages
\usepackage{amsmath}
\usepackage{amssymb}
\usepackage{mathtools}
\usepackage{cancel}
% Formatting and layout
\usepackage[margin=1in]{geometry}
\usepackage{enumitem}
\usepackage{xcolor}
\usepackage{enumitem}

% Allow align environments to break across pages
\allowdisplaybreaks[4]

% Begin document
\begin{document}
\title{Probeklaur}
\author{Victor Minig}
\date{\today}
\maketitle

\section*{A)}
\textbf{Gegeben:}

\[f(x,y) = ke^{-x-2y}I_{(0,\infty)(x)I_{0, \infty}}(y)\]
\subsection*{a1}
\textbf{Gesucht:}\\

Der Wert für $k$ wo $f(x,y)$ tatsächlich eine PDF ist.\\\\
\textbf{Lösung:} \\

Wir suchen also $k$, sodass $\int_{-\infty}^{\infty} f(x,y) dx dy= 1$.

\begin{align*}
    \int_{-\infty}^{\infty} \int_{-\infty}^{\infty} f(x,y) dx dy &= \int_{0}^{\infty} \int_{0}^{\infty} ke^{-x-2y} dx dy \\
    &= k \int_{0}^{\infty} \left[-e^{-x-2y}\right]_0^{\infty} dy \\
    &= k \int_{0}^{\infty} \left[0 - (-e^{-2y}) \right] dy \\
    &= k \left[-\frac{1}{2}e^{-2y}\right]_0^{\infty} \\
    &= k \left[0- (-\frac{1}{2}e^{0})\right] \\
    &= k\frac{1}{2} \\ \\
    k \frac{1}{2} &\overset{!}{=} 1 \\
    \Rightarrow k &= 2  
\end{align*}

\subsection*{a2}
\textbf{Gesucht:} \\ 

Die gemeinsame Ferteilungsfunktion $F(X, Y)$\\
\textbf{Lösung:} \\

\begin{align*}
    F(X,Y) = \int_{0}^{b_y} \int_{0}^{b_x} f(x,y) dx dy 
    &=\int_{0}^{b_y} \int_{0}^{b_x} 2e^{-x-2y} dx dy \\ 
    &= \int_{0}^{b_y} \left[-e^{-x-2y}\right]_0^{b_x} dy \\
    &= \int_{0}^{b_y} \left[-e^{-x-2y}\right]_0^{b_x} dy\\
    &= \int_{0}^{b_y}  -e^{-b_x-2y} + e^{-2y} dy \\
    &= \left[\frac{1}{2} e^{-b_x-2 y}- \frac{1}{2}e^{-2y}\right]_0^{b_y} \\
    &= \frac{1}{2} e^{-b_x - 2b_y} - \frac{1}{2} e^{2b_y} - \frac{1}{2} e^{-bx} + \frac{1}{2}  \\
    &= \frac{1}{2} (e^{-b_x - 2b_y} -e^{-2by} - e^{b_x} + 1) \\
\end{align*}

\subsection*{c)}
\textbf{Gesucht:} \\

Marginalen Dichten von $X$ und $Y$ \\\\
\textbf{Lösung}:

\begin{align*}
    f_y(y) &= \int_{-\infty}^{\infty} f(x,y) dx \\
    &= \int_{-\infty}^{\infty} 2e^{-x-2y} dx \\
    &= \left[-2e^{-x-2y}\right]_0^{\infty} dx \\
    &= 0 - (- 2 e^{-2y}) I_{0, \infty} \\
    &= 2 e^{-2y} I_{0, \infty} \\
    f_x(x) &= \int_{-\infty}^{\infty} f(x,y) dy \\
    &= e^{-x} I_{0 \infty}(x) 
\end{align*}
\subsection*{d} 
\textbf{Gesucht:}

$E(XY)$ \\\\
\textbf{Lösung:}\\

\begin{align*}
    E(XY) &= 2 \int_{0}^{\infty}\int_{0}^{\infty} xy e^{-x-2y} dxdy \\
    &= 2\int_{0}^{\infty} [-xe^{-x-2y}]_0^{\infty} - \int_{0}^{\infty} -e^{-x-2y} dxdy \\
    &= 2\int_{0}^{\infty} \left[0 + 0\right] -\left[e^{-x-2y}\right]_0^{\infty} dy \\
    &= 2\int_{0}^{\infty}-  \left[ 0- e^{-0-2y}\right] dy \\
    &= 2\left[-\frac{1}{2}e^{2y}\right]_0^{\infty}\\
    &= 2(0 + \frac{1}{2} e^{2\cdot 0}) \\
    &= 2 \cdot \frac{1}{2} \\
    &= 1
\end{align*}
\subsection*{e)}
\textbf{Gesucht:} \\

Die bedingte Dichte $f(x|y)$\\\\
\textbf{Lösung:}\\
\begin{align*}
    f(x|y) &= \frac{f(x,y)}{f_y(y)} \\
    &=\frac{2e^{-x-2y}}{2e^{-2y}} \\
    &= \frac{e^{-x} e^{-2y}}{e^{-2y}} \\
    &= e^{-x}
\end{align*}

\subsection*{f)}
\textbf{Gesucht:} \\

Die Wahrscheinlichkeit für $P(x< \frac{1}{2}y, 0 < y < 2)$\\\\
\textbf{Lösung:}
\begin{align*}
    P(x < \frac{1}{2}y, 0 < y < 2) &= \int_{0}^{2} \int_0^{\frac{1}{2}y} f(x,y) dx dy \\
    &= \int_{0}^{2} \left[-2e^{-x-2y}\right]^{\frac{1}{2}y}_0 dy \\
    &= \int_{0}^{2} \left[-2e^{-\frac{1}{2}y - 2y} + 2e^{-2y}\right] dy \\
    &= \left[e^{-\frac{1}{2}y - 2y} - e^{-2y}\right]_0^{2} \\
    &= e^{-1 - 4} - e^{-4} - 1 + 1 \\
    &= e^{-5} -e^{-4}
\end{align*}

\subsection*{g)}
\textbf{Frage:}\\

Sind $X$ und $Y$ stochastisch unabhängig?\\\\
\textbf{Lösung:} \\

Ja, da $f(x|y) = f_x(x)$
\section*{2}
\textbf{Gegeben:} \\
\[f(x) = \frac{1}{b-a}I{[a,b]}(x)\qquad \text{mit }\{(a,b): -\infty < a < b < \infty\}.\]
\subsection*{a)}
\textbf{Zu Zeigen:}\\

Die nichtzentralen Momente sind gegeben durch: 
\[\mu_r'= \frac{b^{r+1}- a^{r+1}}{(b-a)}(r+1)\qquad r=1,2 \ldots\]
\textbf{Lösung:}\\

\textbf{Induktionsanfang:} \\

\begin{align*}
    \text{Für } r = 1\qquad \mu_r' &= \int_{-\infty}^{\infty} x f(x) dx \\
    &= \int_{a}^{b} \frac{x}{a - b} dx \\
    &= \left[\frac{x^2}{2(a-b)}\right]_a^b \\
    &= \frac{b^2}{2 (a-b)} - \frac{a^2}{2 (a-b)} \\
    &= \frac{b^{r+1} - a^{r+1}}{(b-a) (r+1)}
\end{align*}

\textbf{Intduktionsvoraussetzung:}\\

\[\mu_r' = \int_{-\infty}^{\infty}xf(x)dx = \frac{b^{r+1} - a^{r+1}}{(b-a) (r+1)}  \qquad \text{für ein } r \in \mathbb{N}  \]

\textbf{Induktionsschritt:}
\begin{align*}
    \mu_{r+1}' &= \int_{a}^{b} x^{r+1} f(x) dx \\
    &= \left[\frac{x^{(r+1)+1}}{(r+1)+1(b-a)}\right]_a^b \\
    &= \frac{b^{(r+1)+1}}{((r+1)+1)(b-a)} - \frac{a^{(r+1)+1}}{(r+1)+1(b-a)} \\
    &= \frac{b^{(r+1)+1}- a^{(r+1)+1}}{((r+1)+1)(b-a)}
\end{align*}
$\Longrightarrow$ Es gilt also für alle $\forall n \in \mathbb{N}$
\subsection*{b)}
\textbf{Frage:} \\

Ist die gegebene Dichte ein Mitglied der Eponentialfamilie?
\textbf{Lösung:} \\

Nein, da kein $\Theta$ existiert, sodass: \[f(x;\Theta) = \frac{1}{\Theta}e^{-\frac{x}{\Theta}}I_{(0, \infty)} = \frac{1}{b-a} I_{[a,b]}(x)\]

\section*{3} 
\textbf{Gegeben:} \\

X eine bivariat Normalverteilte ZV $\mathcal{N}(\bf{\mu}, \bf{\Sigma})$ mit \[ \bf{\mu} = \begin{pmatrix}
    4 \\
    9
\end{pmatrix} \qquad \text{und} \qquad \bf{\Sigma} = \begin{pmatrix}
    3&1\\
    1&2
\end{pmatrix}\]
\textbf{Gesucht:}\\

Regressionsfunktion von $X_1$ auf $X_2$ und $E(X_1|x_2 = 1)$ \\\\
\textbf{Lösung:} \\

Die Regressionsfunktion lässt sich in unserem Fall darstellen durch $E(X_1|X_2)$ 
\begin{align*}
    E(X_1|X_2) &= \int_{-\infty}^{\infty} x_1 \frac{f(\bf{x})}{f_{x_2}(x_2)}  dx_1\\
    &= 
\end{align*}

\end{document}